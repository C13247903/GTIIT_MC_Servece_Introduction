% !TeX spellcheck = <none>
\documentclass[UTF8,a4paper]{article}
\usepackage[top=2.5cm, bottom=2.5cm, left=2.5cm, right=2.5cm]{geometry}
\usepackage{ctex}
\usepackage{algorithm}
\usepackage{algorithmicx}
\usepackage{algpseudocode}
\usepackage{amsmath}
\usepackage{esint}
\usepackage{graphicx} %插入图片的宏包
\usepackage{float} %设置图片浮动位置的宏包
\usepackage{subfigure} %插入多图时用子图显示的宏包
% \setlength{\parskip}{0.5em}
\title{Minecraft游戏入门,以及如何掌握自行学习的能力}
\author{编纂人:13247903 \and 9DA62C42}
\begin{document}
	\maketitle
	\tableofcontents
	\newpage
	\par 本介绍和指南仅供参考,请访问网络时注意遵守法律法规并识别网站,注意防止诈骗。各个文段作者将标于文段后括号内。
	\section*{前言}
		\par 本教程的创作目的,正如标题所示有两个部分。第一部分是关于Minecraft游戏入门,在这个部分我会给大家介绍一些Minecraft游戏从安装到游玩的操作流程,尤其是在游玩本校服务器时的注意事项。而第二个部分则是希望读者可以掌握在互联网时代下如何使用搜索引擎和一些实用性的网站来进行自行学习的能力。
		\par 这两个目的并不会在教程中明确的做出区分,或者说这并不意味着教程会被分为两部分,一部分叫“Minecraft游戏入门”,另一部分叫“如何自行学习”。这两个话题会融合在一起来讨论。而本教程确实会分为多个不同的部分,具体可以移步目录。
		\par 鉴于阅读这个教程的人很有可能是大一新生,有可能因为学业压力或者家庭原因,在上大学之前并没有过多的使用或深入使用过电脑,或者没有深入游玩Minecraft Java版的经验,所以本教程将会尽量以0基础的前提进行创作,竭尽所能帮助新玩家学习和学会学学习。
		\par 希望您能认真看完本教程,并且将其举一反三地运用到您在互联网生活中的各个方面。
		\par Enjoy your gaming time!
		\begin{flushright}
			(13247903)
		\end{flushright}
		\begin{figure}[H] %H为当前位置,!htb为忽略美学标准,htbp为浮动图形
			\centering %图片居中
			\includegraphics[width=0.3\textwidth]{./Pictures/maodiechongni.png} %插入图片,[]中设置图片大小,{}中是图片文件名
			% \caption{Main name 2} %最终文档中希望显示的图片标题
			\label{Fig.main2} %用于文内引用的标签
		\end{figure}
	\section{如何进入你服}
		\subsection{Java的安装}
			\subsubsection{如何选择Java版本}
				\par 在Java官网得到的安装包版本最高只有8,如需JDK24,JDK21,需要移步oracle→产品→Java并选择你需要的Java版本。
				\par 如果你需要使用本服务器的生电整合包,推荐Java17。
			\subsubsection{对于LauncherX启动器}
				\par 如果你要使用生电整合包,那么不建议使用LauncherX启动器。
				\par 使用LauncherX启动器可以大大简化Java安装难度。打开软件→设置→Java虚拟机设定→下载Java→选择合适的版本。
		\subsection{你服基本信息}
			\subsubsection{版本等基本信息}
				\par 版本:1.20.1,模式:纯净生存。
			\subsubsection{连接方式}
				\par 使用网线连接校园网后在游戏内添加服务器[Github上隐藏ip],或者直接使用连接[Github上隐藏ip]。前者只有在学校内使用校园网才行。
				\par ip地址请移步你服QQ群群公告中查看。
		\subsection{启动器选择}
			\subsubsection{不建议使用的启动器}
				\par 不建议使用官方启动器,LauncherX启动器,因为无法安装整合包,如果无需模组辅助可以考虑(不建议)。
				\par 推荐使用PCL2和HMCL,如何下载请移步互联网。
			\subsubsection{整合包在哪里}
				\par 整合包在群文件里,搜索1.20.1,下载“1.20.1生电整合包.zip”,并且记录下载位置,之后有用。
			\subsubsection{HMCL}
				\par 把下载的HMCL主体放在硬盘某个空白文件夹中,双击运行。
				\par 如何安装整合包:打开软件→左侧选择“版本列表”→安装整合包→导入本地整合包文件→找到整合包文件并安装。
			\subsubsection{PCL2}
				\par 把下载的PCL2主体放在硬盘某个空白文件夹中,双击运行。
				\par 如何安装整合包:打开软件→下载→整合包→安装已有整合包→找到整合包文件并安装。
	\section{服务器内部基本介绍}
		\subsection{末地工业区}
			\subsubsection{大致地理位置}
				\par $(100,\sim,0)$: 刷沙机(末地部分)
				\par $(326,\sim,0)$: 小黑塔
				\par $(119,\sim,-236)$: 树场
				\par $(377,\sim,-241)$: 苔藓骨粉机
				\par $(-146,\sim,-250)$: 刷石机
				\par $(-428,\sim,-423)$: 320熔炉组
				\par $(-189,\sim,-387)$: 农场
				\par $(19,\sim,-412)$: 村民交易所
				\par $(-24,\sim,-566)$: 刷铁机
				\par $(-36,\sim,-757)$: 袭击塔
			\subsubsection{刷沙机(末地部分)}
				\par 注意:刷沙机不可空跑,运行时需要保证假人Sand1, Sand2都在线。
				\par 末地部分的刷沙机主要用于收集及固化混凝土粉末,固化功能需要手动开启。
				\par 重力方块,如铁砧、沙子、混凝土粉末在收集区域的西侧。固化后的混凝土在收集区域的东侧。
				\par 如何启用固化机:在确认刷沙机(主世界部分)只刷混凝土粉末时,在末地部分关闭坐标为(100,52,2)的拉杆,固化机会自动固化混凝土粉末并收集。
				\par 如何关闭固化机:在末地部分打开坐标为(100,52,2)的拉杆即可。
			\subsubsection{}
	\section{常见问题}
		\subsection{整合包相关}
			\subsubsection{屏幕上会显示背包内容}
				\par 此现象是物品栏HUD+导致,按"I"可以关闭。此模组的按键绑定可以在选项→按键控制→按键绑定→Inventory HUD+中修改。
				
\end{document}
